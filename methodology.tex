\section{Methodology}

%\subsection{Classification}

One of the major challenge of classification is to properly define classes. We choose to discretize the position into 2,5-meters square. Since the heading is not constant during navigation and can vary a lot without positional change, we also included the orientation of the camera with 10 degrees increments. The objective is to ensure that the images are consistent within each class to limit unnecessary noise during training. 

This representation of the possible position of the robot is very sparse so to improve the efficency of the training, we will limit the number of classes studied. We also need to avoid having a class too strong against the other during the training, our choice is to keep only the classes that have a number of images of at least 50\% of the highest class image count. For our database, this represents around 1000 images per class with 295 classes giving us 300 000 images in total.

We train an implementation of AlexNet on Caffe. We chose Caffe over other deep learning framework because it was easy to use especially for a classification task. AlexNet is a recognized network used for classification. We didn't change the network structure but since we train it from scratch we tuned the number of images used, the size of the mini-batches and the learning rate. We had to find a trade-off between learning time and precision. We used smaller dataset to test the training and to tune those parameters.

--> Include description of the Network (image ?)

The aim was not to obtain a great classification, but to create good features for each class. This means that we are aiming at obtaining convolutional layers that are well formed and rich enough to give a good representation of the image while staying independent of seasonal and local changes.

\subsubsection{Prototypes}
Representations of an image will be found in the network's convolutional layers. As such, we averaged filter responses for every layer over all images in a class. We can thus minimize the impact of variance over the class images (slight heading and position variations). The average of the filter response for one layer of one image gives us a good representation of this image. It can be compared to the prototype of any classes giving us a metric for the similarity of an image to a given class. 


%\subsection{Regression}
The goal of the regression task is to compute the GPS position corresponding to an image of a given place. This image is completely described by the position of the camera and its orientation. Directly using the position and orientation of the camera as labels would have resulted in inhomogeneous labels. Consequently, it was chosen to use a 4-dimension vector. The first two dimensions are the GPS coordinates of the camera and the other two are the coordinates of the projection on the side of the lake in the direction of the view. To ensure convergence the labels were normalized and re-centered. We will refer to the computed scaling factor as \textit{scale} in the following Results section.

For comparison matters the same Network CaffeNet was used for the regression task. Yet some changes were made on the loss layer which previously used the SoftMax function. It was chosen to use the euclidean loss function in this case where the labels are distances. The creation of the dataset is similar to the one performed for the classification.

The implementation was done with Caffe after some modifications so it could deal with real value labels and multi-label regression.


\subsection{Data preprocessing}


The dataset consists in a bit more than 3'000'000 images collected during 80 surveys between the second half of 2013 up to the end of 2015. We consider two experimental setups : a classification task and a regression task. For the classification task, a label is affected to each image based on the pose of the robot. The pose, consisting in the position (from the GPS) and the heading (from the compass) of the robot, is discretized. The position of the robot is discretized into 2.5 meters squares positioned around the lake on a 350 m by 600 m grid centred on a reference point in the center of the lake\todo{how are the blocks positioned ?}. The heading is discretized in non overlapping angular sectors of 10 degrees. This led to a total of 1'209'600 possible classes.Only a fraction of those possible classes represents images from our dataset and some of the obtained classes were underrepresented. Therefore the dataset was sub-sampled to ensure the balance of the classes. Namely, we kept only the classes with a number of images at least 50\% of the largest class, containing 1750 images. This represents a total of 295 classes with approximately 1'000 images for most classes for a total of 300'000 images on the training set \todo{what is roughly the distribution of the sizes of the classes ? min and max bounds ?} and 5000 images selected randomly for the testing set\todo{how is the dataset split in training and test set ?}. For the regression task, the same set of images was used and the labels were defined from the pose of the robot. One possibility would have been to use the position and heading of the robot as labels but this would imply to define a specific loss taking into account the angular nature of the heading. We rather used an Euclidean loss and therefore defined the labels as a fourth component vector with the position of the robot (from the GPS) and the position of the point 10 meters away from the robot along the optical axis of the camera\todo{is that true?}. One potential drawback of this approach is that the regression problem becomes more complicated than if we were to predict the position and heading since it requires the regressor to predict a specific location along the heading. However it turns out that despite this constraint, the regressor performed reasonably well. Finally, in order to ease the definition of the learning rate and to speed up convergence of learning, the labels to be regressed are normalized and centered.


\subsection{Convolutional neural network architecture and training}

In this study, we used the AlexNet convolutional neural architecture \cite{NIPS2012_4824} trained in Caffe \cite{jia2014caffe}. The network consists in five convolution layers, five pooling layers, seven rectified linear unit layers, two normalisation layers and three fully connected layers\todo{is it true?}. Minor modifications were required to train successfully the AlexNet network. For the classification task, the size of the mini-batches is decreased to 32 samples and learning rate was decreased at a regular rate 10 times throughout training. \todo{how is it changed?}. For the classification task, the output layer of the fully connected part of the architecture uses a softmax transfer function to get a probability distribution over the labels and the loss is the cross-entropy classification loss. For the regression task, a linear output transfer function is considered and the loss is the Euclidean loss. Experimentally, it was required to consider a lower learning rate than AlexNet which otherwise lead to a divergent loss. As we shall see in the result section, several strategies for setting the learning rate are considered. For both the regression and classification problems, the architecture is trained with the default CaffeNet settings, namely stochastic gradient descent\todo{what is the algorithm used?}, a momentum set to 0.9 and a weight decay to 0.0005.


\subsection{Extracting season invariant representations}

Being able to classify an image as belonging to one part of the lake with its viewpoint or to regress from it the pose of the boat is of interest by themselves. However, one of the objectives of the study was also to extract season invariant representations in order to detect the changes of the lake shore. This part of the study was done only from the network trained in the classification task. For every class of the selected dataset, a prototypical image was computing by averaging all the filter responses, at a given depth of the network, of all the images belonging to the considered class. A query image is then propagated through the network up to the depth where the prototypes have been computed, the responses of all the filters at that depth are then averaged and this representation is compared to the computed prototypes. The quality of the computed prototypical images is then assessed from the cosine similarity between the representation of the query image and the prototypes; labelling a new image is then performed by picking the class whose prototype has the largest similarity with the averaged representation of an image.
