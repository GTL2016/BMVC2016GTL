
\begin{abstract}
    This paper focuses on the evaluation of deep convolutional neural networks
    for the analysis of images of natural scenes subjected to large
    seasonal variation as well as significant changes of lighting conditions.
    The context is the development of tools for long-term natural environment
    monitoring with an autonomous mobile robot. 

    We report various experiments conducted on a large dataset consisting of a
    weekly survey of the shore of a small lake over two years using an
    autonomous surface vessel. This dataset is used first in a place
    recognition task framed as a classification problem, then in a pose
    regression task and finally the internal features learned by the
    network are evaluated for their representation power. 

    All our results are based on the Caffe library and default network
    structures where possible.
\end{abstract}
