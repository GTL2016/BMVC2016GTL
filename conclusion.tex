\section{Conclusions}
This paper evaluated the performance of convolutional neural network on place
recognition and pose prediction tasks for natural environment under large
seasonal changes, in the context of a long-term autonomous monitoring problem.
To this end, we presented an original dataset consisting in several million
images taken at weekly interval on the shore of a small lake over two years. 

Water and sky appearance inconsistency, as well as the strong seasonal changes
of vegetation and the weather-dependent lighting conditions proved to be
manageable both for the classification task (70\% precision) and for the pose
regression task (20m standard deviation over 1km of shore line). However, it
turned out that using the standard network architecture did not result in
learning generalizable features leading to a season-invariant representation of
the environment. Turning towards more general network architectures as
in~\cite{radford2016} would probably be appropriate but it turned out that
Caffe was too limited to explore this possibility within this study.
